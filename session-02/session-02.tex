\documentclass[10pt,aspectratio=169]{beamer}

% Use the metropolis theme
\usetheme{metropolis}

% Set the color scheme
\usepackage{xcolor}
\definecolor{mDarkTeal}{HTML}{23373b}
\definecolor{mLightBrown}{HTML}{EB811B}
\definecolor{mLightGreen}{HTML}{14B03D}

% Math packages
\usepackage{amsmath}
\usepackage{amssymb}
\usepackage{mathtools}

% Graphics
\usepackage{graphicx}
\usepackage{tikz}

% Additional packages
\usepackage{booktabs}
\usepackage{multicol}

% Title information
\title{F1010 - Modeling with Differential Equations}
\subtitle{Session 2: First-Order Differential Equations \\ Linear and Separable Equations}
\author{Dr. Juliho Castillo\\julihocc@tec.mx}
\institute{School of Engineering and Sciences\\Academic Department of Sciences}
\date{\today}

\begin{document}

% Title slide
\maketitle

% Table of contents
\begin{frame}{Outline}
    \tableofcontents
\end{frame}

\section{Introduction and Recap}

\begin{frame}{Session Objectives}
    By the end of this session, you will be able to:
    \begin{itemize}
        \item Identify and define first-order linear differential equations.
        \item Solve first-order linear differential equations using the integrating factor method.
        \item Identify and define separable differential equations.
        \item Solve separable differential equations by separating variables and integrating.
    \end{itemize}
\end{frame}

\section{Linear Equations with Variable Coefficients}

\begin{frame}{Definition and Standard Form}
    A \textbf{first-order linear differential equation} can be written in the standard form:
    \begin{equation}
        \frac{dy}{dx} + P(x)y = Q(x)
        \label{eq:linear_de_standard_form}
    \end{equation}
    where $P(x)$ and $Q(x)$ are continuous functions of $x$.
    \vspace{1em}
    \textbf{Key characteristics:}
    \begin{itemize}
        \item The dependent variable $y$ and its derivative $\frac{dy}{dx}$ appear to the first power.
        \item There are no products of $y$ and $\frac{dy}{dx}$.
        \item Coefficients $P(x)$ and $Q(x)$ can be functions of the independent variable $x$.
    \end{itemize}
\end{frame}

\begin{frame}{Method of Solution: Integrating Factor (Part 1)}
    To solve $\frac{dy}{dx} + P(x)y = Q(x)$:
    \begin{enumerate}
        \item Calculate the \textbf{integrating factor}, $\mu(x)$:
        \begin{equation}
            \mu(x) = e^{\int P(x)dx}
            \label{eq:integrating_factor}
        \end{equation}
        \item Multiply the entire standard DE (Equation \ref{eq:linear_de_standard_form}) by $\mu(x)$:
        \begin{equation}
            \mu(x)\frac{dy}{dx} + \mu(x)P(x)y = \mu(x)Q(x)
        \end{equation}
        The left side becomes the derivative of a product: $\frac{d}{dx}[\mu(x)y]$.
        \begin{equation}
            \frac{d}{dx}[\mu(x)y] = \mu(x)Q(x)
        \end{equation}
    \end{enumerate}
\end{frame}

\begin{frame}{Method of Solution: Integrating Factor (Part 2)}
    Continuing the solution for $\frac{dy}{dx} + P(x)y = Q(x)$:
    \begin{enumerate}
        \setcounter{enumi}{2} % Continue numbering from previous frame
        \item Integrate both sides with respect to $x$:
        \begin{equation}
            \mu(x)y = \int \mu(x)Q(x)dx + C
        \end{equation}
        \item Solve for $y(x)$:
        \begin{equation}
            y(x) = \frac{1}{\mu(x)} \left( \int \mu(x)Q(x)dx + C \right)
            \label{eq:linear_de_solution}
        \end{equation}
    \end{enumerate}
\end{frame}

\begin{frame}{Example: Linear DE (Problem Setup)}
    Solve the differential equation: $x\frac{dy}{dx} - 2y = x^2$ for $x > 0$.
    \vspace{1em}
    \textbf{Step 1: Standard Form}
    \begin{itemize}
        \item Divide by $x$ to match the form $\frac{dy}{dx} + P(x)y = Q(x)$.
        \item $\frac{dy}{dx} - \frac{2}{x}y = x$.
        \item Identify $P(x) = -\frac{2}{x}$ and $Q(x) = x$.
    \end{itemize}
    \vspace{1em}
    \textbf{Step 2: Integrating Factor $\mu(x)$}
    \begin{itemize}
        \item $\mu(x) = e^{\int P(x)dx} = e^{\int -\frac{2}{x}dx}$.
        \item $\int -\frac{2}{x}dx = -2\ln|x| = \ln|x|^{-2} = \ln x^{-2}$ (since $x > 0$).
        \item So, $\mu(x) = e^{\ln x^{-2}} = x^{-2} = \frac{1}{x^2}$.
    \end{itemize}
\end{frame}

\begin{frame}{Example: Linear DE (Solution Steps)}
    Continuing with $x\frac{dy}{dx} - 2y = x^2$, where $\mu(x) = \frac{1}{x^2}$.
    \vspace{1em}
    \textbf{Step 3: Multiply by $\mu(x)$ and Integrate}
    \begin{itemize}
        \item Multiply the standard form by $\mu(x)$: $\frac{1}{x^2}\left(\frac{dy}{dx} - \frac{2}{x}y\right) = \frac{1}{x^2} \cdot x$.
        \item This simplifies to $\frac{1}{x^2}\frac{dy}{dx} - \frac{2}{x^3}y = \frac{1}{x}$.
        \item The left side is $\frac{d}{dx}\left[\frac{1}{x^2}y\right]$.
        \item So, $\frac{d}{dx}\left[\frac{1}{x^2}y\right] = \frac{1}{x}$.
        \item Integrate: $\frac{1}{x^2}y = \int \frac{1}{x}dx = \ln|x| + C = \ln x + C$ (since $x > 0$).
    \end{itemize}
    \vspace{1em}
    \textbf{Step 4: Solve for $y(x)$}
    \begin{itemize}
        \item $y(x) = x^2(\ln x + C)$.
    \end{itemize}
\end{frame}

\section{Separable Equations}

\begin{frame}{Definition and Standard Form}
    A first-order differential equation is \textbf{separable} if it can be written in the form:
    \begin{equation}
        M(x)dx + N(y)dy = 0
        \label{eq:separable_form1}
    \end{equation}
    or, equivalently, if it can be expressed as:
    \begin{equation}
        \frac{dy}{dx} = f(x)g(y)
        \label{eq:separable_form2}
    \end{equation}
    The key is that all terms involving $x$ can be grouped with $dx$, and all terms involving $y$ can be grouped with $dy$.
\end{frame}

\begin{frame}{Method of Solution: Separation and Integration}
    To solve a separable equation of the form $\frac{dy}{dx} = f(x)g(y)$:
    \begin{enumerate}
        \item \textbf{Separate variables}: If $g(y) \neq 0$, rewrite as:
        \begin{equation}
            \frac{1}{g(y)}dy = f(x)dx
        \end{equation}
        \item \textbf{Integrate both sides} with respect to their respective variables:
        \begin{equation}
            \int \frac{1}{g(y)}dy = \int f(x)dx + C
            \label{eq:separable_solution}
        \end{equation}
        where $C$ is the constant of integration.
        \item \textbf{Solve for $y(x)$} if possible, or leave the solution in implicit form.
    \end{enumerate}
\end{frame}

\begin{frame}{Example: Separable DE -- Problem and Step 1}
    Solve the differential equation: $\frac{dy}{dx} = \frac{x^2}{y^2}$.
    \vspace{1em}
    \textbf{Step 1: Separate Variables}
    $y^2 dy = x^2 dx$.
\end{frame}

\begin{frame}{Example: Separable DE -- Step 2 (Integration)}
    \textbf{Step 2: Integrate Both Sides}
    $\int y^2 dy = \int x^2 dx + C$.
    $\frac{y^3}{3} = \frac{x^3}{3} + C_1$.
\end{frame}

\begin{frame}{Example: Separable DE -- Step 3 (Solve for $y(x)$)}
    \textbf{Step 3: Solve for $y(x)$ (optional form)}
    $y^3 = x^3 + 3C_1$. Let $C = 3C_1$.
    $y^3 = x^3 + C$.
    $y(x) = \sqrt[3]{x^3 + C}$.
\end{frame}

\section{Summary and Next Steps}

\begin{frame}{Key Takeaways}
    \begin{itemize}
        \item \textbf{Linear First-Order DEs} ($\frac{dy}{dx} + P(x)y = Q(x)$):
        \begin{itemize}
            \item Solved using an integrating factor $\mu(x) = e^{\int P(x)dx}$.
            \item General solution: $y(x) = \frac{1}{\mu(x)} \left( \int \mu(x)Q(x)dx + C \right)$.
        \end{itemize}
        \item \textbf{Separable DEs} ($\frac{dy}{dx} = f(x)g(y)$):
        \begin{itemize}
            \item Solved by separating variables: $\frac{1}{g(y)}dy = f(x)dx$.
            \item Integrate both sides: $\int \frac{1}{g(y)}dy = \int f(x)dx + C$.
        \end{itemize}
    \end{itemize}
\end{frame}

\begin{frame}{Looking Ahead: Session 3}
    \textbf{Next Session (Session 3): Modeling with First-Order Equations}
    \begin{itemize}
        \item Applications of linear and separable equations.
        \item Constructing mathematical models for physical phenomena:
        \begin{itemize}
            \item Population dynamics.
            \item Mixing problems.
            \item Radioactive decay.
            \item Newton's law of cooling.
        \end{itemize}
        \item Setting up and solving these models.
    \end{itemize}
    \vspace{1em}
    \textbf{Preparation:}
    \begin{itemize}
        \item Review today's methods for solving linear and separable DEs.
        \item Consider how rates of change are described in physical systems.
    \end{itemize}
\end{frame}

\begin{frame}[standout]
    Questions?
    \vspace{1cm}
    Practice these techniques!
\end{frame}

\end{document}
