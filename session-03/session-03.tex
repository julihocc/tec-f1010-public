\documentclass[10pt,aspectratio=169]{beamer}

% Use the metropolis theme
\usetheme{metropolis}

% Math packages
\usepackage{amsmath}
\usepackage{amssymb}
\usepackage{mathtools}

% Graphics
\usepackage{graphicx}
\usepackage{tikz}

% Additional packages
\usepackage{booktabs}
\usepackage{multicol}

% Problem environment for structured presentation
\usepackage{tcolorbox}
\definecolor{problemBg}{HTML}{f0f8ff}
\definecolor{problemBorder}{HTML}{4682b4}

% Counter for problems
\newcounter{problemcounter}

\newtcolorbox{problembox}{
    colback=problemBg,
    colframe=problemBorder,
    boxrule=2pt,
    arc=5pt,
    left=10pt,
    right=10pt,
    top=10pt,
    bottom=10pt,
    fonttitle=\bfseries,
    title=Problem \number\numexpr\value{problemcounter}+1\relax,
    before upper={\stepcounter{problemcounter}}
}

% Title information
\title{F1010 - Modeling with Differential Equations}
\subtitle{Session 3: Modeling with First-Order Equations and Physical Applications}
\author{Dr. Juliho Castillo \and \texttt{julihocc@tec.mx}}
\institute{Tec de Monterrey}
\date{\today}

\begin{document}

% Reset problem counter for this session
\setcounter{problemcounter}{0}

% Title slide
\maketitle

% Table of contents
\begin{frame}{Outline}
    \tableofcontents
\end{frame}

%----------------------
\section{Session Objectives}
\begin{frame}{Session Objectives}
    By the end of this session, you will be able to:
    \begin{itemize}
        \item Apply mathematical modeling principles to first-order differential equations.
        \item Construct and analyze population dynamics models (exponential and logistic growth).
        \item Model and solve Newton's law of cooling/heating problems.
        \item Set up and solve mixing problems and compartmental analysis.
        \item Apply first-order DEs to RC circuit applications.
        \item Interpret the physical meaning of solutions.
    \end{itemize}
\end{frame}

%----------------------
\section{Introduction to Mathematical Modeling}
\begin{frame}{What is Mathematical Modeling?}
    \begin{itemize}
        \item Mathematical modeling is the process of representing real-world phenomena using mathematical equations and concepts.
        \item Differential equations are a key tool for describing how quantities change over time in physical, biological, and engineering systems.
        \item The modeling process involves:
        \begin{enumerate}
            \item Identifying variables and parameters
            \item Making simplifying assumptions
            \item Formulating the governing equations
            \item Solving and interpreting the results
        \end{enumerate}
    \end{itemize}
\end{frame}

%----------------------
\section{Population Dynamics}
\begin{frame}{Exponential Growth Model}
    \textbf{Model:} $\frac{dP}{dt} = rP$
    \begin{itemize}
        \item $P(t)$: population at time $t$
        \item $r$: constant growth rate
        \item \textbf{Solution:} $P(t) = P_0 e^{rt}$
        \item \textbf{Assumptions:} Unlimited resources, no deaths, immigration, or emigration
    \end{itemize}
\end{frame}

\begin{frame}{Exponential Growth Model - Problem}
    \begin{problembox}
        \textbf{Scenario:} A bacterial culture starts with 100 bacteria and grows at a rate of 25\% per hour.
        
        \textbf{Given Information:}
        \begin{itemize}
            \item Initial population: $P_0 = 100$ bacteria
            \item Growth rate: $r = 0.25$ per hour
        \end{itemize}
        
        \textbf{Questions to Answer:}
        \begin{enumerate}
            \item What is the population after 4 hours?
            \item How long does it take for the population to reach 1000 bacteria?
        \end{enumerate}
        
        \textbf{Setup:} We need to apply the exponential growth model $\frac{dP}{dt} = rP$
    \end{problembox}
\end{frame}

\begin{frame}{Exponential Growth Model - Solution}
    \textbf{Model Application:} $P(t) = P_0 e^{rt} = 100e^{0.25t}$
    
    \textbf{Solution to Question 1:}
    \begin{itemize}
        \item Population after 4 hours: 
        \item $P(4) = 100e^{0.25 \cdot 4} = 100e^1 \approx 100 \times 2.718 = 272$ bacteria
    \end{itemize}
    
    \textbf{Solution to Question 2:}
    \begin{itemize}
        \item Time to reach 1000 bacteria:
        \item $1000 = 100e^{0.25t}$
        \item $10 = e^{0.25t}$
        \item $\ln(10) = 0.25t$
        \item $t = \frac{\ln(10)}{0.25} \approx \frac{2.303}{0.25} = 9.2$ hours
    \end{itemize}
\end{frame}

\begin{frame}{Logistic Growth Model}
    \textbf{Model:} $\frac{dP}{dt} = rP\left(1-\frac{P}{K}\right)$
    \begin{itemize}
        \item $K$: carrying capacity
        \item \textbf{Solution:} $P(t) = \frac{KP_0 e^{rt}}{K + P_0 (e^{rt} - 1)}$
        \item \textbf{Assumptions:} Limited resources, population cannot exceed $K$
    \end{itemize}
\end{frame}

\begin{frame}{Logistic Growth Model - Problem}
    \begin{problembox}
        \textbf{Scenario:} A population of bacteria in a petri dish follows a logistic growth model.
        
        \textbf{Given Information:}
        \begin{itemize}
            \item Initial population: $P_0 = 1000$ bacteria
            \item Carrying capacity: $K = 50,000$ bacteria
            \item Population after 1 hour: $P(1) = 2000$ bacteria
        \end{itemize}
        
        \textbf{Questions to Answer:}
        \begin{enumerate}
            \item Find the growth rate parameter $r$
            \item Determine the complete logistic growth equation $P(t)$
            \item What will the population be after 3 hours?
        \end{enumerate}
        
        \textbf{Setup:} We need to apply the logistic growth model $\frac{dP}{dt} = rP\left(1-\frac{P}{K}\right)$
    \end{problembox}
\end{frame}

\begin{frame}{Logistic Growth Model - Solution}
    \textbf{Model Application:} $P(t) = \frac{KP_0 e^{rt}}{K + P_0 (e^{rt} - 1)}$
    
    \textbf{Solution to Question 1 (Find $r$):}
    \begin{itemize}
        \item Substitute known values at $t=1$: $2000 = \frac{50000 \cdot 1000 e^{r}}{50000 + 1000 (e^{r} - 1)}$
        \item Simplifying: $2000 = \frac{50000000 e^{r}}{49000 + 1000 e^{r}}$
        \item Cross multiply and solve: $e^r \approx 2.085$
        \item Therefore: $r = \ln(2.085) \approx 0.7348$
    \end{itemize}
    
    \textbf{Solution to Question 2 (Complete equation):}
    $P(t) = \frac{50000 \cdot 1000 e^{0.7348t}}{50000 + 1000 (e^{0.7348t} - 1)}$
    
    \textbf{Solution to Question 3 (Population after 3 hours):}
    $P(3) = \frac{50000000 e^{2.2044}}{49000 + 1000 e^{2.2044}} \approx 4762$ bacteria
\end{frame}

%----------------------
\section{Newton's Law of Cooling/Heating}
\begin{frame}{Newton's Law of Cooling/Heating}
    \textbf{Model:} $\frac{dT}{dt} = -k(T - T_{env})$
    \begin{itemize}
        \item $T(t)$: temperature of the object
        \item $T_{env}$: ambient temperature
        \item $k$: positive constant
        \item \textbf{Solution:} $T(t) = T_{env} + (T_0 - T_{env})e^{-kt}$
    \end{itemize}
\end{frame}

\begin{frame}{Newton's Law of Cooling - Problem}
    \begin{problembox}
        \textbf{Scenario:} A cup of coffee is initially $90^{\circ}\text{C}$ and is left in a room with an ambient temperature of $20^{\circ}\text{C}$.
        
        \textbf{Given Information:}
        \begin{itemize}
            \item Initial temperature: $T_0 = 90^{\circ}\text{C}$
            \item Ambient temperature: $T_{env} = 20^{\circ}\text{C}$
            \item Temperature after 5 minutes: $T(5) = 60^{\circ}\text{C}$
        \end{itemize}
        
        \textbf{Questions to Answer:}
        \begin{enumerate}
            \item Find the cooling constant $k$
            \item Determine the complete temperature equation $T(t)$
            \item What will the temperature be after 10 minutes?
        \end{enumerate}
        
        \textbf{Setup:} We need to apply Newton's Law of Cooling $\frac{dT}{dt} = -k(T - T_{env})$
    \end{problembox}
\end{frame}

\begin{frame}{Newton's Law of Cooling - Solution}
    \textbf{Model Application:} $T(t) = T_{env} + (T_0 - T_{env})e^{-kt} = 20 + 70e^{-kt}$
    
    \textbf{Solution to Question 1 (Find $k$):}
    \begin{itemize}
        \item Substitute known values at $t=5$: $60 = 20 + 70e^{-5k}$
        \item Simplifying: $40 = 70e^{-5k}$
        \item Solve: $e^{-5k} = \frac{4}{7}$
        \item Therefore: $k = -\frac{1}{5}\ln\left(\frac{4}{7}\right) \approx 0.1118$
    \end{itemize}
    
    \textbf{Solution to Question 2 (Complete equation):}
    $T(t) = 20 + 70e^{-0.1118t}$
    
    \textbf{Solution to Question 3 (Temperature after 10 minutes):}
    $T(10) = 20 + 70e^{-0.1118 \cdot 10} = 20 + 70e^{-1.118} \approx 20 + 70(0.3269) \approx 42.88^{\circ}\text{C}$
\end{frame}

%----------------------
\section{Mixing Problems and Compartmental Analysis}
\begin{frame}{Mixing Problem: Basic Setup}
    \begin{itemize}
        \item Consider a tank with volume $V$ containing a solution.
        \item Inflow: solution with concentration $c_{in}$ enters at rate $r_{in}$.
        \item Outflow: well-mixed solution leaves at rate $r_{out}$.
        \item Let $Q(t)$ be the amount of substance at time $t$.
        \item \textbf{Model:} $\frac{dQ}{dt} = r_{in}c_{in} - r_{out}\frac{Q}{V}$
    \end{itemize}
\end{frame}

%-----------------------------------------------------------------------
\subsection{Mixing Problems}
%-----------------------------------------------------------------------
\begin{frame}{Mixing Problems}
    \begin{itemize}
        \item Involve a substance (e.g., salt) dissolving in a fluid (e.g., water) within a container.
        \item The fluid enters and leaves the container at certain rates.
        \item The concentration of the substance changes over time.
        \item \textbf{Goal:} Model the amount of substance in the container at time $t$.
        \item \textbf{DE Form:} $\frac{dA}{dt} = (\text{Rate In}) - (\text{Rate Out})$
        \item Rate In = (flow rate of liquid in) $\times$ (concentration of substance in inflow)
        \item Rate Out = (flow rate of liquid out) $\times$ (concentration of substance in outflow)
        \item Concentration in outflow = $A(t) / V(t)$, where $V(t)$ is the volume of liquid at time $t$.
    \end{itemize}
\end{frame}

\begin{frame}{Mixing Problem - Problem}
    \begin{problembox}
        \textbf{Scenario:} A tank initially contains 1000 L of brine with 10 kg of dissolved salt. Pure water enters the tank at a rate of 10 L/min, and the solution drains from the tank at the same rate.
        
        \textbf{Given Information:}
        \begin{itemize}
            \item Initial volume: $V = 1000$ L (constant since inflow = outflow)
            \item Initial salt amount: $A(0) = 10$ kg
            \item Inflow rate: 10 L/min of pure water (0 kg/L concentration)
            \item Outflow rate: 10 L/min of well-mixed solution
        \end{itemize}
        
        \textbf{Questions to Answer:}
        \begin{enumerate}
            \item Set up the differential equation for salt amount $A(t)$
            \item Solve the differential equation
            \item How much salt is in the tank after 20 minutes?
        \end{enumerate}
        
        \textbf{Setup:} Use the mixing model $\frac{dA}{dt} = (\text{Rate In}) - (\text{Rate Out})$
    \end{problembox}
\end{frame}

\begin{frame}{Mixing Problem - Solution}
    \textbf{Solution to Question 1 (Set up DE):}
    \begin{itemize}
        \item Rate In: (10 L/min) $\times$ (0 kg/L) = 0 kg/min
        \item Rate Out: (10 L/min) $\times$ ($A(t)/1000$ kg/L) = $A(t)/100$ kg/min
        \item DE: $\frac{dA}{dt} = 0 - \frac{A(t)}{100} = -\frac{A}{100}$
    \end{itemize}
    
    \textbf{Solution to Question 2 (Solve DE):}
    \begin{itemize}
        \item Separable: $\frac{dA}{A} = -\frac{1}{100} dt$
        \item Integrate: $\ln|A| = -\frac{t}{100} + C$
        \item General solution: $A(t) = Ke^{-t/100}$
        \item Using $A(0) = 10$: $K = 10$
        \item Model: $A(t) = 10e^{-t/100}$
    \end{itemize}
    
    \textbf{Solution to Question 3 (Salt after 20 minutes):}
    $A(20) = 10e^{-20/100} = 10e^{-0.2} \approx 10(0.8187) \approx 8.187$ kg
\end{frame}

%-----------------------------------------------------------------------
\subsection{RC Circuit Applications}
%-----------------------------------------------------------------------
\begin{frame}{RC Circuit Applications}
    \begin{itemize}
        \item Series circuit: resistor $R$, capacitor $C$, voltage source $V_0$.
        \item \textbf{Model:} $RC\frac{dQ}{dt} + Q = V_0C$
        \item \textbf{Solution:} $Q(t) = V_0C(1 - e^{-t/(RC)})$
    \end{itemize}
\end{frame}

%-----------------------------------------------------------------------
\begin{frame}{RC Circuit - Problem}
    \begin{problembox}
        \textbf{Scenario:} A resistor of $R=5\,\Omega$ and a capacitor of $C=0.02\,\mathrm{F}$ are connected in series with a battery of $E=100\,\mathrm{V}$.
        
        \textbf{Given Information:}
        \begin{itemize}
            \item Resistance: $R = 5\,\Omega$
            \item Capacitance: $C = 0.02\,\mathrm{F}$
            \item Voltage source: $E = 100\,\mathrm{V}$ (constant)
            \item Initial charge: $Q(0) = 0\,\mathrm{C}$
        \end{itemize}
        
        \textbf{Questions to Answer:}
        \begin{enumerate}
            \item Set up the differential equation for charge $Q(t)$
            \item Solve the differential equation using integrating factor method
            \item Find the steady-state charge as $t \to \infty$
        \end{enumerate}
        
        \textbf{Setup:} Use the RC circuit model $R\frac{dQ}{dt} + \frac{1}{C}Q = E(t)$
    \end{problembox}
\end{frame}

\begin{frame}{RC Circuit - Solution}
    \textbf{Solution to Question 1 (Set up DE):}
    \begin{itemize}
        \item Substitute values: $5\frac{dQ}{dt} + \frac{1}{0.02}Q = 100$
        \item Simplify: $5\frac{dQ}{dt} + 50Q = 100$
        \item Standard form: $\frac{dQ}{dt} + 10Q = 20$
    \end{itemize}
    
    \textbf{Solution to Question 2 (Solve using integrating factor):}
    \begin{itemize}
        \item Integrating factor: $I(t) = e^{\int 10 dt} = e^{10t}$
        \item Multiply DE by IF: $e^{10t}\frac{dQ}{dt} + 10e^{10t}Q = 20e^{10t}$
        \item Left side becomes: $\frac{d}{dt}(Qe^{10t}) = 20e^{10t}$
        \item Integrate: $Qe^{10t} = 2e^{10t} + K$
        \item General solution: $Q(t) = 2 + Ke^{-10t}$
        \item Using $Q(0) = 0$: $K = -2$
        \item Model: $Q(t) = 2(1 - e^{-10t})$
    \end{itemize}
    
    \textbf{Solution to Question 3 (Steady-state):}
    As $t \to \infty$, $Q(t) \to 2\,\mathrm{C}$ (steady-state charge)
\end{frame}

%----------------------
\section{Physical Interpretation of Solutions}
\begin{frame}{Interpreting Solutions}
    \begin{itemize}
        \item Solutions to first-order DEs describe how quantities evolve over time.
        \item The form of the solution reflects the underlying physical process (e.g., exponential growth/decay, approach to equilibrium).
        \item Always check units and physical plausibility.
        \item Use initial conditions to determine specific solutions.
    \end{itemize}
\end{frame}

%----------------------
\section{Summary and Next Steps}
\begin{frame}{Key Takeaways}
    \begin{itemize}
        \item Mathematical modeling translates real-world problems into solvable equations.
        \item First-order DEs are widely used in population dynamics, cooling/heating, mixing, and circuits.
        \item Understanding assumptions and interpreting solutions is crucial.
    \end{itemize}
\end{frame}

\begin{frame}{Looking Ahead}
    \textbf{Next Session:} Second-Order Differential Equations and Oscillations
    \begin{itemize}
        \item Introduction to oscillatory motion
        \item Linear differential operators
        \item Fundamental solutions of homogeneous equations
    \end{itemize}
\end{frame}

\begin{frame}[standout]
    Questions?\newline
    Practice modeling and solving real-world problems!
\end{frame}

\end{document}
