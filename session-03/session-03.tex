\documentclass[10pt,aspectratio=169]{beamer}

% Use the metropolis theme
\usetheme{metropolis}

% Math packages
\usepackage{amsmath}
\usepackage{amssymb}
\usepackage{mathtools}

% Graphics
\usepackage{graphicx}
\usepackage{tikz}

% Additional packages
\usepackage{booktabs}
\usepackage{multicol}

% Title information
\title{F1010 - Modeling with Differential Equations}
\subtitle{Session 3: Modeling with First-Order Equations and Physical Applications}
\author{Dr. Juliho Castillo\\ \\texttt{julihocc@tec.mx}}
\institute{Tec de Monterrey}
\date{\today}

\begin{document}

% Title slide
\maketitle

% Table of contents
\begin{frame}{Outline}
    \tableofcontents
\end{frame}

%----------------------
\section{Session Objectives}
\begin{frame}{Session Objectives}
    By the end of this session, you will be able to:
    \begin{itemize}
        \item Apply mathematical modeling principles to first-order differential equations.
        \item Construct and analyze population dynamics models (exponential and logistic growth).
        \item Model and solve Newton's law of cooling/heating problems.
        \item Set up and solve mixing problems and compartmental analysis.
        \item Apply first-order DEs to RC circuit applications.
        \item Interpret the physical meaning of solutions.
    \end{itemize}
\end{frame}

%----------------------
\section{Introduction to Mathematical Modeling}
\begin{frame}{What is Mathematical Modeling?}
    \begin{itemize}
        \item Mathematical modeling is the process of representing real-world phenomena using mathematical equations and concepts.
        \item Differential equations are a key tool for describing how quantities change over time in physical, biological, and engineering systems.
        \item The modeling process involves:
        \begin{enumerate}
            \item Identifying variables and parameters
            \item Making simplifying assumptions
            \item Formulating the governing equations
            \item Solving and interpreting the results
        \end{enumerate}
    \end{itemize}
\end{frame}

%----------------------
\section{Population Dynamics}
\begin{frame}{Exponential Growth Model}
    \textbf{Model:} $\frac{dP}{dt} = rP$
    \begin{itemize}
        \item $P(t)$: population at time $t$
        \item $r$: constant growth rate
        \item \textbf{Solution:} $P(t) = P_0 e^{rt}$
        \item \textbf{Assumptions:} Unlimited resources, no deaths, immigration, or emigration
    \end{itemize}
\end{frame}

\begin{frame}{Logistic Growth Model}
    \textbf{Model:} $\frac{dP}{dt} = rP\left(1-\frac{P}{K}\right)$
    \begin{itemize}
        \item $K$: carrying capacity
        \item \textbf{Solution:} $P(t) = \frac{KP_0 e^{rt}}{K + P_0 (e^{rt} - 1)}$
        \item \textbf{Assumptions:} Limited resources, population cannot exceed $K$
    \end{itemize}
\end{frame}

\begin{frame}{Example: Logistic Growth}
    A population of bacteria in a petri dish follows a logistic growth model. Initially, there are 1000 bacteria. The carrying capacity of the dish is estimated to be 50,000 bacteria. After 1 hour, the population has grown to 2000.
    \begin{itemize}
        \item \textbf{Problem:} Determine the logistic growth equation $P(t)$ for this population.
        \item \textbf{Given:} $P_0 = 1000$, $K = 50000$, $P(1) = 2000$.
        \item \textbf{Recall solution form:} $P(t) = \frac{KP_0 e^{rt}}{K + P_0 (e^{rt} - 1)}$
        \item We need to find $r$. Substitute $t=1, P(1)=2000$:
        \item $2000 = \frac{50000 \cdot 1000 e^{r}}{50000 + 1000 (e^{r} - 1)}$
        \item Solving for $e^r$ gives $e^r \approx 2.085$. So, $r = \ln(2.085) \approx 0.7348$.
        \item \textbf{Model:} $P(t) = \frac{50000 \cdot 1000 e^{0.7348t}}{50000 + 1000 (e^{0.7348t} - 1)}$
    \end{itemize}
\end{frame}

%----------------------
\section{Newton's Law of Cooling/Heating}
\begin{frame}{Newton's Law of Cooling/Heating}
    \textbf{Model:} $\frac{dT}{dt} = -k(T - T_{env})$
    \begin{itemize}
        \item $T(t)$: temperature of the object
        \item $T_{env}$: ambient temperature
        \item $k$: positive constant
        \item \textbf{Solution:} $T(t) = T_{env} + (T_0 - T_{env})e^{-kt}$
    \end{itemize}
\end{frame}

\begin{frame}{Example: Newton's Law of Cooling}
    A cup of coffee is initially $90^{\circ}\text{C}$. It is left in a room with an ambient temperature of $20^{\circ}\text{C}$. After 5 minutes, the coffee has cooled to $60^{\circ}\text{C}$.
    \begin{itemize}
        \item \textbf{Problem:} Temp. after 10 mins?
        \item \textbf{Given:} $T_0 = 90^{\circ}\text{C}$, $T_{env} = 20^{\circ}\text{C}$, $T(5) = 60^{\circ}\text{C}$.
        \item \textbf{Solution form:} $T(t) = T_{env} + (T_0 - T_{env})e^{-kt}$
        \item Substitute: $60 = 20 + (90 - 20)e^{-5k}$
        \item Solve for $k$: $40 = 70e^{-5k} \implies e^{-5k} = \\frac{4}{7}$
        \item So, $-5k = \\ln(\\frac{4}{7}) \implies k = -\\frac{1}{5}\\ln(\\frac{4}{7}) \\approx 0.1118$.
        \item \textbf{Model:} $T(t) = 20 + 70e^{-0.1118t}$.
        \item \textbf{Temp. at $t=10$ min:} $T(10) = 20 + 70e^{-0.1118 \cdot 10} \approx 20 + 70e^{-1.118} \approx 20 + 70(0.3269) \approx 42.88^{\circ}\text{C}$.
    \end{itemize}
\end{frame}

%----------------------
\section{Mixing Problems and Compartmental Analysis}
\begin{frame}{Mixing Problem: Basic Setup}
    \begin{itemize}
        \item Consider a tank with volume $V$ containing a solution.
        \item Inflow: solution with concentration $c_{in}$ enters at rate $r_{in}$.
        \item Outflow: well-mixed solution leaves at rate $r_{out}$.
        \item Let $Q(t)$ be the amount of substance at time $t$.
        \item \textbf{Model:} $\frac{dQ}{dt} = r_{in}c_{in} - r_{out}\frac{Q}{V}$
    \end{itemize}
\end{frame}

%-----------------------------------------------------------------------
\subsection{Mixing Problems}
%-----------------------------------------------------------------------
\begin{frame}{Mixing Problems}
    \begin{itemize}
        \item Involve a substance (e.g., salt) dissolving in a fluid (e.g., water) within a container.
        \item The fluid enters and leaves the container at certain rates.
        \item The concentration of the substance changes over time.
        \item \textbf{Goal:} Model the amount of substance in the container at time $t$.
        \item \textbf{DE Form:} $\\frac{dA}{dt} = (Rate In) - (Rate Out)$
        \item Rate In = (flow rate of liquid in) $\\times$ (concentration of substance in inflow)
        \item Rate Out = (flow rate of liquid out) $\\times$ (concentration of substance in outflow)
        \item Concentration in outflow = $A(t) / V(t)$, where $V(t)$ is the volume of liquid at time $t$.
    \end{itemize}
\end{frame}

\begin{frame}{Example: Mixing Problem}
    A tank initially contains 1000 L of brine with 10 kg of dissolved salt. Pure water enters the tank at a rate of 10 L/min. The solution is kept thoroughly mixed and drains from the tank at the same rate.
    \begin{itemize}
        \item \textbf{Problem:} How much salt is in the tank after 20 minutes?
        \item \textbf{Let $A(t)$ be the amount of salt (kg) at time $t$ (min).}
        \item $A(0) = 10$ kg.
        \item Volume $V(t) = 1000$ L (constant, since inflow rate = outflow rate).
        \item \textbf{Rate In:} (10 L/min) $\\times$ (0 kg/L) = 0 kg/min (pure water).
        \item \textbf{Rate Out:} (10 L/min) $\\times$ ($A(t)/1000$ kg/L) = $A(t)/100$ kg/min.
        \item \textbf{DE:} $\\frac{dA}{dt} = 0 - \\frac{A(t)}{100} = -\\frac{A}{100}$.
        \item This is a separable DE: $\\frac{dA}{A} = -\\frac{1}{100} dt$.
        \item Integrate: $\\ln|A| = -\\frac{t}{100} + C \implies A(t) = Ke^{-t/100}$.
        \item Use $A(0)=10$: $10 = Ke^0 \implies K=10$.
        \item \textbf{Model:} $A(t) = 10e^{-t/100}$.
        \item \textbf{Salt at $t=20$ min:} $A(20) = 10e^{-20/100} = 10e^{-0.2} \\approx 10(0.8187) \\approx 8.187$ kg.
    \end{itemize}
\end{frame}

%-----------------------------------------------------------------------
\subsection{RC Circuit Applications}
%-----------------------------------------------------------------------
\begin{frame}{RC Circuit Applications}
    \begin{itemize}
        \item Series circuit: resistor $R$, capacitor $C$, voltage source $V_0$.
        \item \textbf{Model:} $RC\frac{dQ}{dt} + Q = V_0C$
        \item \textbf{Solution:} $Q(t) = V_0C(1 - e^{-t/RC})$
    \end{itemize}
\end{frame}

%-----------------------------------------------------------------------
\begin{frame}{Example: RC Circuit}
    A resistor of $R=5 \\Omega$ and a capacitor of $C=0.02 F$ are connected in series with a battery of $E=100 V$. The initial charge on the capacitor is $Q_0 = 0 C$.
    \begin{itemize}
        \item \textbf{Problem:} Find the charge $Q(t)$ on the capacitor at time $t$.
        \item \textbf{Given:} $R=5 \\Omega$, $C=0.02 F$, $E(t)=100 V$, $Q(0)=0 C$.
        \item \textbf{DE for RC Circuit:} $R\\frac{dQ}{dt} + \\frac{1}{C}Q = E(t)$.
        \item Substitute values: $5\\frac{dQ}{dt} + \\frac{1}{0.02}Q = 100 \implies 5\\frac{dQ}{dt} + 50Q = 100$.
        \item Simplify: $\\frac{dQ}{dt} + 10Q = 20$. This is a linear first-order DE.
        \item \textbf{Integrating Factor (IF):} $I(t) = e^{\\int 10 dt} = e^{10t}$.
        \item Multiply by IF: $e^{10t}\\frac{dQ}{dt} + 10e^{10t}Q = 20e^{10t}$.
        \item The left side is $\\frac{d}{dt}(Qe^{10t})$.
        \item So, $\\frac{d}{dt}(Qe^{10t}) = 20e^{10t}$.
        \item Integrate: $Qe^{10t} = \\int 20e^{10t} dt = 20 \\frac{e^{10t}}{10} + K = 2e^{10t} + K$.
        \item Solve for $Q(t)$: $Q(t) = 2 + Ke^{-10t}$.
        \item Use $Q(0)=0$: $0 = 2 + Ke^0 \implies K = -2$.
        \item \textbf{Model:} $Q(t) = 2 - 2e^{-10t}$.
        \item As $t \\to \\infty$, $Q(t) \\to 2 C$ (steady-state charge).
    \end{itemize}
\end{frame}

%----------------------
\section{Physical Interpretation of Solutions}
\begin{frame}{Interpreting Solutions}
    \begin{itemize}
        \item Solutions to first-order DEs describe how quantities evolve over time.
        \item The form of the solution reflects the underlying physical process (e.g., exponential growth/decay, approach to equilibrium).
        \item Always check units and physical plausibility.
        \item Use initial conditions to determine specific solutions.
    \end{itemize}
\end{frame}

%----------------------
\section{Summary and Next Steps}
\begin{frame}{Key Takeaways}
    \begin{itemize}
        \item Mathematical modeling translates real-world problems into solvable equations.
        \item First-order DEs are widely used in population dynamics, cooling/heating, mixing, and circuits.
        \item Understanding assumptions and interpreting solutions is crucial.
    \end{itemize}
\end{frame}

\begin{frame}{Looking Ahead}
    \textbf{Next Session:} Second-Order Differential Equations and Oscillations
    \begin{itemize}
        \item Introduction to oscillatory motion
        \item Linear differential operators
        \item Fundamental solutions of homogeneous equations
    \end{itemize}
\end{frame}

\begin{frame}[standout]
    Questions?\newline
    Practice modeling and solving real-world problems!
\end{frame}

\end{document}
