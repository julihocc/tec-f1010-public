\documentclass[10pt,aspectratio=169]{beamer}

% Use the metropolis theme
\usetheme{metropolis}

% Math packages
\usepackage{amsmath}
\usepackage{amssymb}
\usepackage{mathtools}

% Graphics
\usepackage{graphicx}
\usepackage{tikz}

% Additional packages
\usepackage{booktabs}
\usepackage{multicol}

% Title information
\title{F1010 - Modeling with Differential Equations}
\subtitle{Session 3: Modeling with First-Order Equations and Physical Applications}
\author{Dr. Juliho Castillo\\julihocc@tec.mx}
\institute{Tec de Monterrey}
\date{\today}

\begin{document}

% Title slide
\maketitle

% Table of contents
\begin{frame}{Outline}
    \tableofcontents
\end{frame}

%----------------------
\section{Session Objectives}
\begin{frame}{Session Objectives}
    By the end of this session, you will be able to:
    \begin{itemize}
        \item Apply mathematical modeling principles to first-order differential equations.
        \item Construct and analyze population dynamics models (exponential and logistic growth).
        \item Model and solve Newton's law of cooling/heating problems.
        \item Set up and solve mixing problems and compartmental analysis.
        \item Apply first-order DEs to RC circuit applications.
        \item Interpret the physical meaning of solutions.
    \end{itemize}
\end{frame}

%----------------------
\section{Introduction to Mathematical Modeling}
\begin{frame}{What is Mathematical Modeling?}
    \begin{itemize}
        \item Mathematical modeling is the process of representing real-world phenomena using mathematical equations and concepts.
        \item Differential equations are a key tool for describing how quantities change over time in physical, biological, and engineering systems.
        \item The modeling process involves:
        \begin{enumerate}
            \item Identifying variables and parameters
            \item Making simplifying assumptions
            \item Formulating the governing equations
            \item Solving and interpreting the results
        \end{enumerate}
    \end{itemize}
\end{frame}

%----------------------
\section{Population Dynamics}
\begin{frame}{Exponential Growth Model}
    \textbf{Model:} $\frac{dP}{dt} = rP$
    \begin{itemize}
        \item $P(t)$: population at time $t$
        \item $r$: constant growth rate
        \item \textbf{Solution:} $P(t) = P_0 e^{rt}$
        \item \textbf{Assumptions:} Unlimited resources, no deaths, immigration, or emigration
    \end{itemize}
\end{frame}

\begin{frame}{Logistic Growth Model}
    \textbf{Model:} $\frac{dP}{dt} = rP\left(1-\frac{P}{K}\right)$
    \begin{itemize}
        \item $K$: carrying capacity
        \item \textbf{Solution:} $P(t) = \frac{KP_0 e^{rt}}{K + P_0 (e^{rt} - 1)}$
        \item \textbf{Assumptions:} Limited resources, population cannot exceed $K$
    \end{itemize}
\end{frame}

%----------------------
\section{Newton's Law of Cooling/Heating}
\begin{frame}{Newton's Law of Cooling/Heating}
    \textbf{Model:} $\frac{dT}{dt} = -k(T - T_{env})$
    \begin{itemize}
        \item $T(t)$: temperature of the object
        \item $T_{env}$: ambient temperature
        \item $k$: positive constant
        \item \textbf{Solution:} $T(t) = T_{env} + (T_0 - T_{env})e^{-kt}$
    \end{itemize}
\end{frame}

%----------------------
\section{Mixing Problems and Compartmental Analysis}
\begin{frame}{Mixing Problem: Basic Setup}
    \begin{itemize}
        \item Consider a tank with volume $V$ containing a solution.
        \item Inflow: solution with concentration $c_{in}$ enters at rate $r_{in}$.
        \item Outflow: well-mixed solution leaves at rate $r_{out}$.
        \item Let $Q(t)$ be the amount of substance at time $t$.
        \item \textbf{Model:} $\frac{dQ}{dt} = r_{in}c_{in} - r_{out}\frac{Q}{V}$
    \end{itemize}
\end{frame}

%----------------------
\section{RC Circuit Applications}
\begin{frame}{RC Circuit: Charging a Capacitor}
    \begin{itemize}
        \item Series circuit: resistor $R$, capacitor $C$, voltage source $V_0$.
        \item \textbf{Model:} $RC\frac{dQ}{dt} + Q = V_0C$
        \item \textbf{Solution:} $Q(t) = V_0C(1 - e^{-t/RC})$
    \end{itemize}
\end{frame}

%----------------------
\section{Physical Interpretation of Solutions}
\begin{frame}{Interpreting Solutions}
    \begin{itemize}
        \item Solutions to first-order DEs describe how quantities evolve over time.
        \item The form of the solution reflects the underlying physical process (e.g., exponential growth/decay, approach to equilibrium).
        \item Always check units and physical plausibility.
        \item Use initial conditions to determine specific solutions.
    \end{itemize}
\end{frame}

%----------------------
\section{Summary and Next Steps}
\begin{frame}{Key Takeaways}
    \begin{itemize}
        \item Mathematical modeling translates real-world problems into solvable equations.
        \item First-order DEs are widely used in population dynamics, cooling/heating, mixing, and circuits.
        \item Understanding assumptions and interpreting solutions is crucial.
    \end{itemize}
\end{frame}

\begin{frame}{Looking Ahead}
    \textbf{Next Session:} Second-Order Differential Equations and Oscillations
    \begin{itemize}
        \item Introduction to oscillatory motion
        \item Linear differential operators
        \item Fundamental solutions of homogeneous equations
    \end{itemize}
\end{frame}

\begin{frame}[standout]
    Questions?\newline
    Practice modeling and solving real-world problems!
\end{frame}

\end{document}
