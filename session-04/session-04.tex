% F1010 - Modeling with Differential Equations
% Session 4: Practice - First-Order Differential Equations
% Dr. Juliho Castillo
% Tec de Monterrey

\documentclass[10pt,aspectratio=169]{beamer}

% Use the metropolis theme
\usetheme{metropolis}

% Math packages
\usepackage{amsmath}
\usepackage{amssymb}
\usepackage{mathtools}

% Graphics
\usepackage{graphicx}
\usepackage{tikz}

% Additional packages
\usepackage{booktabs}
\usepackage{multicol}

% Custom commands for academic style
\newcommand{\concept}[1]{\textbf{#1}}
\newcommand{\formula}[1]{\textit{#1}}
\newcommand{\emphasis}[1]{\textit{#1}}

% Title information
\title{F1010 - Modeling with Differential Equations}
\subtitle{Session 4: Problem-Solving Workshop \\ First-Order Differential Equations}
\author{Dr. Juliho Castillo\\\texttt{julihocc@tec.mx}}
\institute{Tec de Monterrey}
\date{\today}

\begin{document}

% Title slide
\maketitle

% Table of contents
\begin{frame}{Outline}
    \tableofcontents
\end{frame}

%----------------------
\section{Session Objectives}
\begin{frame}{Session Objectives}
    By the end of this workshop, you will be able to:
    \begin{itemize}
        \item Apply solution techniques for linear and separable first-order DEs to a variety of problems.
        \item Develop strategies for setting up mathematical models from problem descriptions.
        \item Analyze and interpret solutions in the context of given scenarios.
        \item Enhance problem-solving skills through guided practice and case studies.
    \end{itemize}
\end{frame}

%----------------------
\section{Warm-up: Quick Review Problems}
\begin{frame}{Warm-up: Linear Equation}
    \concept{Problem 1:} Solve the initial value problem:
    \formula{\[ \frac{dy}{dt} + 2ty = 2te^{-t^2}, \quad y(0) = 1 \]}
    \begin{itemize}
        \item Identify the type of equation.
        \item Determine the integrating factor.
        \item Find the general solution.
        \item Apply the initial condition.
    \end{itemize}
\end{frame}

\begin{frame}{Warm-up: Separable Equation}
    \concept{Problem 2:} Find the general solution for:
    \formula{\[ \frac{dy}{dx} = \frac{xy+3x-y-3}{xy-2x+4y-8} \]}
    \begin{itemize}
        \item Factor the numerator and denominator to separate variables.
        \item Integrate both sides.
        \item Express the solution (implicitly or explicitly).
    \end{itemize}
\end{frame}

%----------------------
\section{Modeling Exercises}
\begin{frame}{Modeling Exercise 1: Tank Mixing Problem}
    \concept{Scenario:} A tank initially contains 100L of brine with 10kg of dissolved salt. Brine containing 0.5 kg/L of salt enters the tank at a rate of 4 L/min. The solution is kept thoroughly mixed and drains from the tank at a rate of 4 L/min.
    \begin{enumerate}
        \item Set up a differential equation for the amount of salt $A(t)$ in the tank at time $t$.
        \item Solve the differential equation.
        \item How much salt is in the tank after 20 minutes?
        \item What is the long-term concentration of salt in the tank?
    \end{enumerate}
\end{frame}

\begin{frame}{Modeling Exercise 2: Newton's Law of Cooling}
    \concept{Scenario:} A metal object at $80^\circ C$ is placed in a room where the temperature is $25^\circ C$. After 10 minutes, the object's temperature is $60^\circ C$.
    \begin{enumerate}
        \item Formulate the differential equation based on Newton's Law of Cooling.
        \item Solve the DE to find the temperature $T(t)$ of the object at time $t$.
        \item What will be the temperature of the object after 30 minutes?
        \item When will the object reach $30^\circ C$?
    \end{enumerate}
\end{frame}

%----------------------
\section{Case Studies (Group Work)}
\begin{frame}{Case Study 1: Population Dynamics - Logistic Growth with Harvesting}
    \concept{Problem:} Consider a fish population $P(t)$ that follows logistic growth with a carrying capacity $K$ and intrinsic growth rate $r$. Suppose fish are harvested at a constant rate $H$.
    \formula{\[ \frac{dP}{dt} = rP\left(1-\frac{P}{K}\right) - H \]}
    \begin{itemize}
        \item Discuss the equilibrium solutions and their stability for different values of $H$.
        \item What is the maximum sustainable harvest rate?
        \item Sketch phase lines for different scenarios.
    \end{itemize}
    \emphasis{This problem requires qualitative analysis and interpretation.}
\end{frame}

\begin{frame}{Case Study 2: Chemical Reaction Kinetics}
    \concept{Problem:} Two chemicals A and B react to form a product C. The rate of formation of C is proportional to the product of the remaining amounts of A and B. Let $x(t)$ be the amount of C formed at time $t$. If initially there are $a_0$ units of A and $b_0$ units of B, and $x$ units of C are formed from $x_A$ units of A and $x_B$ units of B according to stoichiometry.
    \begin{itemize}
        \item Set up the differential equation for $x(t)$. (Consider stoichiometry, e.g., $A+B \to C$)
        \item Solve the differential equation assuming $a_0 \neq b_0$.
        \item What happens if $a_0 = b_0$?
        \item Discuss the limiting amount of product C.
    \end{itemize}
\end{frame}

%----------------------
\section{Wrap-up and Next Steps}
\begin{frame}{Key Learnings from Workshop}
    \begin{itemize}
        \item Reinforce understanding of solution methods for first-order DEs.
        \item Practice translating word problems into mathematical models.
        \item Importance of identifying assumptions in modeling.
        \item Gained experience in analyzing and interpreting solutions.
    \end{itemize}
\end{frame}

\begin{frame}{Looking Ahead: Session 5}
    \textbf{Next Session (Session 5): Second-Order Differential Equations}
    \begin{itemize}
        \item Introduction to oscillations.
        \item Linear differential operators.
        \item Fundamental solutions of homogeneous equations.
    \end{itemize}
    \emphasis{Ensure you are comfortable with first-order DEs before moving on.}
\end{frame}

\begin{frame}[standout]
    Questions on today's problems? \\
    Continue practicing modeling and solving DEs!
\end{frame}

\end{document}
