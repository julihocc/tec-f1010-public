% F1010 - Modeling with Differential Equations
% Session 4: Practice - First-Order Differential Equations
% Dr. Juliho Castillo
% Tec de Monterrey

\documentclass[10pt,aspectratio=169]{beamer}

% Use the metropolis theme
\usetheme{metropolis}

% Math packages
\usepackage{amsmath}
\usepackage{amssymb}
\usepackage{mathtools}

% Graphics
\usepackage{graphicx}
\usepackage{tikz}

% Additional packages
\usepackage{booktabs}
\usepackage{multicol}

% Custom commands for academic style
\newcommand{\concept}[1]{\textbf{#1}}
\newcommand{\formula}[1]{\textit{#1}}
\newcommand{\emphasis}[1]{\textit{#1}}

% Problem environment for structured presentation
\usepackage{tcolorbox}
\definecolor{problemBg}{HTML}{f0f8ff}
\definecolor{problemBorder}{HTML}{4682b4}

% Counter for problems
\newcounter{problemcounter}

\newtcolorbox{problembox}{
    colback=problemBg,
    colframe=problemBorder,
    boxrule=2pt,
    arc=5pt,
    left=10pt,
    right=10pt,
    top=10pt,
    bottom=10pt,
    fonttitle=\bfseries,
    title=Problem \number\numexpr\value{problemcounter}+1\relax,
    before upper={\stepcounter{problemcounter}}
}

\newtcolorbox{solutionbox}{
    colback=white,
    colframe=green!50!black,
    boxrule=2pt,
    arc=5pt,
    left=10pt,
    right=10pt,
    top=10pt,
    bottom=10pt,
    fonttitle=\bfseries,
    title=Solution Approach
}

% Title information
\title{F1010 - Modeling with Differential Equations}
\subtitle{Session 4: Problem-Solving Workshop \\ First-Order Differential Equations}
\author{Dr. Juliho Castillo \\ \texttt{julihocc@tec.mx}}
\institute{Tec de Monterrey}
\date{\today}

\begin{document}

% Reset problem counter for this session
\setcounter{problemcounter}{0}

% Title slide
\maketitle

% Table of contents
\begin{frame}{Outline}
    \tableofcontents
\end{frame}

%----------------------
\section{Session Objectives}
\begin{frame}{Session Objectives}
    By the end of this workshop, you will be able to:
    \begin{itemize}
        \item Apply solution techniques for linear and separable first-order DEs to a variety of problems.
        \item Develop strategies for setting up mathematical models from problem descriptions.
        \item Analyze and interpret solutions in the context of given scenarios.
        \item Enhance problem-solving skills through guided practice and case studies.
    \end{itemize}
\end{frame}

%----------------------
\section{Warm-up: Quick Review Problems}
\begin{frame}{Warm-up: Linear Equation}
    \begin{problembox}
        Solve the initial value problem:
        \formula{\[ \frac{dy}{dt} + 2ty = 2te^{-t^2}, \quad y(0) = 1 \]}
    \end{problembox}
\end{frame}

\begin{frame}{Warm-up: Linear Equation - Solution}
    \begin{solutionbox}
        \begin{itemize}
            \item Identify the type of equation (linear first-order)
            \item Determine the integrating factor: $\mu(t) = e^{\int 2t \, dt} = e^{t^2}$
            \item Find the general solution using the integrating factor
            \item Apply the initial condition to find the particular solution
        \end{itemize}
    \end{solutionbox}
\end{frame}

\begin{frame}{Warm-up: Separable Equation}
    \begin{problembox}
        Find the general solution for:
        \formula{\[ \frac{dy}{dx} = \frac{x}{y} \]}
    \end{problembox}
\end{frame}

\begin{frame}{Warm-up: Separable Equation - Solution}
    \begin{solutionbox}
        \begin{itemize}
            \item Separate variables: $y \, dy = x \, dx$
            \item Integrate both sides: $\int y \, dy = \int x \, dx$
            \item Result: $\frac{y^2}{2} = \frac{x^2}{2} + C$
            \item Express the solution: $y^2 - x^2 = C$ (family of hyperbolas)
        \end{itemize}
    \end{solutionbox}
\end{frame}

%----------------------
\section{Modeling Exercises}
\begin{frame}{Tank Mixing Problem}
    \begin{problembox}
        A tank initially contains 100L of brine with 10kg of dissolved salt. Brine containing 0.5 kg/L of salt enters the tank at a rate of 4 L/min. The solution is kept thoroughly mixed and drains from the tank at a rate of 4 L/min.
        \begin{enumerate}
            \item Set up a differential equation for the amount of salt $A(t)$ in the tank at time $t$.
            \item Solve the differential equation.
            \item How much salt is in the tank after 20 minutes?
            \item What is the long-term concentration of salt in the tank?
        \end{enumerate}
    \end{problembox}
\end{frame}

\begin{frame}{Tank Mixing Problem - Solution Approach}
    \begin{solutionbox}
        \textbf{Key Concepts:}
        \begin{itemize}
            \item Rate of change = Rate in - Rate out
            \item Concentration = Amount/Volume  
            \item Linear first-order equation solution
            \item Long-term behavior analysis
        \end{itemize}
    \end{solutionbox}
\end{frame}

\begin{frame}{Newton's Law of Cooling}
    \begin{problembox}
        A metal object at $80^\circ C$ is placed in a room where the temperature is $25^\circ C$. After 10 minutes, the object's temperature is $60^\circ C$.
        \begin{enumerate}
            \item Write the differential equation and solve for $T(t)$.
            \item Find the temperature after 30 minutes.
        \end{enumerate}
    \end{problembox}
\end{frame}

\begin{frame}{Newton's Law of Cooling - Solution Approach}
    \begin{solutionbox}
        \textbf{Key Concepts:}
        \begin{itemize}
            \item Newton's cooling law: $\frac{dT}{dt} = -k(T - T_{\text{ambient}})$
            \item Exponential decay toward ambient temperature
            \item Parameter determination from data
            \item Temperature prediction
        \end{itemize}
    \end{solutionbox}
\end{frame}

%----------------------
\section{Additional Practice Problems}
\begin{frame}{Growth with Limited Resources}
    \begin{problembox}
        A bacteria population $P(t)$ grows according to:
        \formula{\[ \frac{dP}{dt} = 0.5P\left(1-\frac{P}{1000}\right) \]}
        Given $P(0) = 50$:
        \begin{enumerate}
            \item Solve the differential equation.
            \item Find the population after 5 hours.
            \item What is the carrying capacity?
        \end{enumerate}
    \end{problembox}
\end{frame}

\begin{frame}{Growth with Limited Resources - Solution Approach}
    \begin{solutionbox}
        \textbf{Solution Framework:}
        \begin{itemize}
            \item Recognize as logistic growth model
            \item Use separation of variables or standard logistic solution
            \item Interpret parameters: growth rate and carrying capacity
            \item Population behavior analysis
        \end{itemize}
    \end{solutionbox}
\end{frame}

\begin{frame}{Simple Decay Process}
    \begin{problembox}
        A radioactive substance decays at a rate proportional to the amount present. If 20\% decays in 10 years:
        \begin{enumerate}
            \item Set up and solve the differential equation.
            \item Find the half-life of the substance.
            \item How much remains after 30 years?
        \end{enumerate}
    \end{problembox}
\end{frame}

\begin{frame}{Simple Decay Process - Solution Approach}
    \begin{solutionbox}
        \textbf{Solution Framework:}
        \begin{itemize}
            \item Exponential decay model: $\frac{dN}{dt} = -\lambda N$
            \item Determine decay constant from given information
            \item Calculate half-life: $t_{1/2} = \frac{\ln(2)}{\lambda}$
            \item Long-term decay analysis
        \end{itemize}
    \end{solutionbox}
\end{frame}

%----------------------
\section{Wrap-up and Next Steps}
\begin{frame}{Key Learnings from Workshop}
    Today we practiced:
    \begin{itemize}
        \item Solving linear and separable first-order differential equations
        \item Setting up mathematical models from practical scenarios
        \item Applying initial conditions and interpreting solutions
        \item Working with exponential growth, decay, and logistic models
    \end{itemize}
    \emphasis{Focus on understanding the connection between problem context and mathematical models.}
\end{frame}

\begin{frame}{Looking Ahead: Session 5}
    \textbf{Next Session (Session 5): Second-Order Differential Equations}
    \begin{itemize}
        \item Introduction to oscillations.
        \item Linear differential operators.
        \item Fundamental solutions of homogeneous equations.
    \end{itemize}
    \emphasis{Ensure you are comfortable with first-order DEs before moving on.}
\end{frame}

\begin{frame}[standout]
    Questions on today's problems? \\
    Continue practicing modeling and solving DEs!
\end{frame}

\end{document}
